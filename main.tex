\documentclass[12pt, a4paper]{report}
\usepackage[top=3cm,left=3cm,right=2cm,bottom=2cm]{geometry}
\linespread{1.3}
\setlength{\parindent}{1.25cm}
\usepackage{indentfirst}
\usepackage[utf8]{inputenc}
\usepackage[brazil]{babel}
\usepackage{amsmath}
\usepackage{amsthm}
\usepackage{amsfonts}
\usepackage{amssymb}
\usepackage{graphicx}
\usepackage{color}
\usepackage{multicol}
\usepackage[normalem]{ulem}
\usepackage{wrapfig}
\usepackage{caption}
\usepackage{fancybox}
\usepackage[pdfstartview=FitH]{hyperref}
\usepackage{subfigure}
\bibliographystyle{plain}

\graphicspath{{Figuras/}}

\renewcommand{\theenumii}{\alph{enumii}}
\DeclareMathOperator{\sen}{sen}
\DeclareMathOperator{\tg}{tg}
\DeclareMathOperator{\arctg}{arctg}
\DeclareMathOperator{\cotg}{cotg}
\DeclareMathOperator{\agm}{agm}

\newtheorem{thm}{Teorema}[section]
\newtheorem{dfn}{Definição}[section]
\newtheorem{prob}{Problema}[section]
\newtheorem{cor}{Corolário}[section]
\newtheorem{prop}{Proposição}[section]
\newtheorem{lem}{Lema} [section]

\newcounter{contar}


\begin{document}

%---------- CAPA -------------

\pagestyle{empty}
\hbox {
\includegraphics[scale=0.18]{furgGM.png}
\hspace{10cm}
\includegraphics[scale=0.08]{logo_variao.jpg}}


\begin{center}
\sc{\large{Universidade Federal do Rio Grande - FURG}} \\
\sc{\large{Instituto de Matemática Estatística e Física - IMEF}} \\
\sc{\small{Mestrado em Física}} \\
\sc{\small{Dissertação de Mestrado}} \\

\vspace{4cm}

\sc{\Large{Título da Dissertação}}

\vspace{4.5cm}

\sc{\Large{Leonardo de Albernaz Ferreira}}

\vspace{5.5cm}

\textbf{Rio Grande - Rio Grande do Sul} \\
Dezembro de 2016
\end{center}


%---------- FOLHA DE ROSTO -------------
\newpage
\begin{center}
\sc{\Large{Título da Dissertação}}

\vspace{4cm}

\large{Leonardo de Albernaz Ferreira}
\end{center}

\vspace{4cm}

\begin{flushright}
\begin{minipage}{8.6cm}
Dissertação de Mestrado apresentada à \mbox{Comissão} 
Acadêmica Institucional do PPGFís como 
requisito parcial para obtenção 
do título de Mestre em \mbox{Física}.

\vspace{0.5cm}
\textbf{Orientador}: Prof. Dr. Fabricio Ferrari

\end{minipage}
\end{flushright}
 
\vspace{8cm}


\begin{center}
\textbf{Rio Grande - Rio Grande do Sul} \\
Dezembro de 2016
\end{center}


%---------- BANCA EXAMINADORA -------------
\newpage
\begin{center}
\sc{\Large{Título da Dissertação}}

\vspace{2.2cm}

\large{Leonardo de Albernaz Ferreira}
\end{center}

\vspace{2.2cm}

\begin{flushright}
\begin{minipage}{8.6cm} 
Dissertação de Mestrado apresentada à \mbox{Comissão}
Acadêmica Institucional do PPGFís como requisito
parcial para obtenção do título de Mestre em
Matemática
\end{minipage}
\end{flushright}
 
\vspace{1cm}
\begin{center}
\Large \textbf{Banca Examinadora:}
\end{center}
\vspace{1.5cm}

\begin{flushright}
\begin{minipage}[l]{12cm}
\begin{center}
\uline{\hspace{10.5cm}} \\
Prof. Dr. Fabricio Ferrari (Orientador) \\ FURG \\
\vspace{1cm}
\uline{\hspace{10.5cm}} \\
Prof. Dr. Fulano \\ UFLN \\
\vspace{1cm}
\uline{\hspace{10.5cm}} \\
Prof. Dr. Ciclano \\ UFLN

\end{center}
\end{minipage}
\end{flushright}

%-----------Dedicatória----------------
\newpage
\vspace*{21cm}
\begin{flushright}
\textit{À minha família}
\end{flushright}


%------------Agradecimentos------------
\newpage
\chapter*{Agradecimentos}
\thispagestyle{empty}

Agradeço a...

%------------Citação-------------------
\newpage
\vspace*{20cm}
\begin{flushright}
\begin{minipage}{7cm}
\begin{flushright}
\textit{
"A Matemática não mente. Mente quem faz mau uso dela". \\
Albert Einstein}
\end{flushright}
\end{minipage}
\end{flushright}


%--------------Resumo-------------------
\newpage
\chapter*{Resumo}
\thispagestyle{empty}



%-------------Abstract------------------
\newpage
\chapter*{Abstract}
\thispagestyle{empty}


%-------------Índice--------------------
\newpage
\tableofcontents
\thispagestyle{empty}


%-------------Introdução----------------
\chapter*{Introdução}
\pagestyle{myheadings}
\setcounter{page}{1}
\addcontentsline{toc}{chapter}{Introdução}



%-------------Capítulo 1-----------------
\chapter{Capítulo 1}
	

\section{Seção 1}



%-------------Capítulo 2-----------------
\chapter{Probabilidade e Aprendizado de Máquina}
	

\section{Seção 1}


%-------------Capítulo 3-----------------
\chapter{Capítulo 3}
	

\section{Seção 1}




\nocite{Lucas1, Lucas2, Bonola}


%-------------Bibliografia------------------
\newpage
\renewcommand{\refname}{Referências Bibliográficas}
\addcontentsline{toc}{chapter}{Referências Bibliográficas}
\bibliography{Bibliografia}


\end{document}
